\documentclass{article}

\usepackage{amsmath, amssymb}

\begin{document}
\title{EECS 476: PS2}
\author{James Starkman, jas497}
\date{2016-02-03}
\maketitle
%\setlength{\columnseprule}{0.4pt}.

Let $R$ be the radius of the robot (assumed circular).  Since STDR
appears to be entirely deterministic and devoid of measurement error
(by default, anyways), there is no need for any error terms.  Now, for
each LIDAR ping $i$, let $r_i$ and $\theta_i$ be, respectively, the
radius and angle of that ping.

We wish to ensure a clear tunnel ahead of the robot that is at least
as wide as the robot is, for at least $d=$ \verb|MIN_SAFE_DISTANCE| in
front.  This corresponds to the region
$\{(x,y) \ |\ x\in[0,d], y\in[-R,R]\}$ in the robot's rectilinear
coordinate system (where $+x$ is forwards).  If any laser endpoint is
in this region, we sound the alarm.  The $x$ and $y$ values of each
laser endpoint are computed from their given polar form, thus:

\begin{equation*}
  \begin{array}{rl}
    x_i &= r_i\times\cos(\theta_i) \\
    y_i &= r_i\times\sin(\theta_i) \\
  \end{array}
\end{equation*}

Note that, since the walls of the STDR map/arena/maze/thing appear to
be rather rather fuzzy, and since numerical integration has errors, an
error term has been added to $R$ to avoid close calls.  This value has
been empirically chosen as 0.1 meters for use with the provided
reactive\_commander file.

\end{document}
